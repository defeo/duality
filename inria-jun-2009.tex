\documentclass[10pt,handout]{beamer}

\usepackage[english]{babel}
\usepackage[utf8]{inputenc}
\usepackage{amsmath,amssymb,amsthm,latexsym}
\usepackage{mathrsfs}
\usepackage[all]{xypic}
\usepackage{tikz}
\usetikzlibrary{arrows,shapes,snakes,automata}

\newcommand{\cat}[1]{\mathscr{#1}}
\newcommand{\lcat}[1]{\mathbf{#1}}
\newcommand{\C}{\cat{C}}
\newcommand{\D}{\cat{D}}
\renewcommand{\L}{\cat{L}}
\newcommand{\comp}{\circ}
\newcommand{\size}[1]{\lVert#1\rVert}
\newcommand{\sizein}[1]{\size{#1}^\mathrm{in}}
\newcommand{\sizeout}[1]{\size{#1}_\mathrm{out}}
\DeclareMathOperator{\ob}{ob}
\DeclareMathOperator{\Hom}{Hom}
\DeclareMathOperator{\id}{id}
\DeclareMathOperator{\Id}{Id}
\newcommand{\N}{\mathbb{N}}
\newcommand{\Z}{\mathbb{Z}}
\newcommand{\Complex}{\mathbb{C}}
\newcommand{\R}{\mathbb{R}}
\newcommand{\K}{\mathbb{K}}
\newcommand{\ra}{\rightarrow}
\newcommand{\la}{\leftarrow}
\DeclareMathOperator{\Time}{t}
\DeclareMathOperator{\Space}{s}
\DeclareMathOperator{\coTime}{co-t}
\DeclareMathOperator{\coSpace}{co-s}
\DeclareMathOperator{\Par}{Par}
\newcommand{\computes}{\vdash}


\mode<presentation>{%
%%  \usetheme[]{Madrid}
  \usefonttheme[onlymath]{serif}
}


\title{Transposition Principle}
\subtitle{where categories meet computer algebra}
\author{L.~De~Feo}
\institute[TANC]{Projet TANC, LIX, École Polytechnique}
\date[INRIA Saclay, Juin 17, 2008]{Journée des doctorants INRIA Saclay\\Orsay, June 17, 2008}
\setbeamerfont{title}{size=\Huge}
\setbeamerfont{subtitle}{size=\Large}

\begin{document}

\begin{frame}
  \titlepage
\end{frame}


\begin{frame}
  \frametitle{Tellegen's Principle}

  \Large
  \begin{center}
    ``From every \emph{linear algorithm} computing a
    linear application we can deduce another \emph{linear algorithm}
    computing the transpose application using \emph{about} the same
    space and time resources.''
  \end{center}

  \vfill

  \begin{center}
    What's so special about transposition ?
  \end{center}
\end{frame}

\section{Background}

\begin{frame}
  \frametitle{History, motivations}
  
  \begin{block}{History}
    \begin{itemize}
    \item Originally discovered by \alert{Tellegen (1950)},
      \alert{Bordewijk (1956)} for \emph{electrical network theory}
      and by \alert{Kalman (1960)} for \emph{control theory};
    \item Graph-theoretic approach by \alert{Fettweis (1971)} for
      \emph{digital filters};
    \item \alert{Fiduccia (1972)}: transposition of \emph{bilinear
      algorithms};
    \item Special case of reverse mode in \emph{automatic differentiation}:
      \alert{Baur \& Strassen (1983)};
    \item In \emph{computer algebra}, popularized by \alert{Shoup},
      \alert{von zur Gathen}, \alert{Kaltofen},\dots
    \item \alert{[Bostan, Lecerf, Schost 2003]} improve algorithms for
      polynomial evaluation.
    \end{itemize}
  \end{block}
  
  \begin{block}{Motivations}
    \begin{itemize}
    \item Existence result in \emph{complexity theory};
    \item \emph{Code transformation} technique;
    \item Improve $M^T \Leftrightarrow$ Improve $M$;
    \item {\bf Divides by $2$ the number of algorithms yet to be discovered.}
    \end{itemize}
  \end{block}
\end{frame}


\begin{frame}
  \frametitle{Classical proofs}

  \begin{block}{Linear algebra}
    $M$ computed as a sequence of \emph{simple} linear applications $U_i$
    \begin{equation*}
      M(v) = U_1\circ U_2\circ\cdots\circ U_n(v)
      \qquad\Leftrightarrow\qquad
      M^T(v) = U_n^T\circ\cdots\circ U_2^T\circ U_1^T
    \end{equation*}
  \end{block}

  \begin{block}{Graph-theoretic approach}
    \begin{columns}
      \begin{column}{0.6\textwidth}
        \centering
        \begin{tikzpicture}
          \tikzstyle{node}=[circle,thick,draw=black,minimum size=4mm]
          
          \begin{scope}
            \node(x1){$x_1$};
            \node(x2)[right of=x1]{$x_2$};
            \node(x3)[right of=x2]{$x_3$};
            
            \node[node](d1)[below of=x2]{}
            edge(x2);
            \node[node](d2)[below of=x3]{}
            edge(x3);
            
            \node[node](p1)[below of=d1,xshift=-6mm]{$+$}
            edge(x1)
            edge(d1);
            \node[node](m1)[below of=d2]{$*_3$}
            edge(d2);
            
            \node[node](p2)[below of=p1]{$+$}
            edge(p1)
            edge(d2);
            \node[node](p3)[right of=p2]{$+$}
            edge(d1)
            edge(m1);
            
            \node(y1)[below of=p2]{$y_1$}
            edge(p2);
            \node(y2)[below of=p3]{$y_2$}
            edge(p3);
          \end{scope}
          
          \begin{scope}[xshift=3.2cm, yshift=-0.23\textheight]
            \node(x){\Huge $\leftrightarrow$};
          \end{scope}
          
          \begin{scope}[xshift=4cm, yshift=-0.425\textheight]
            \node(x1){$x_1'$};
            \node(x2)[right of=x1]{$x_2'$};
            \node(x3)[right of=x2]{$x_3'$};
            
            \node[node](d1)[above of=x2]{$+$}
            edge(x2);
            \node[node](d2)[above of=x3]{$+$}
            edge(x3);
            
            \node[node](p1)[above of=d1,xshift=-5mm]{}
            edge(x1)
            edge(d1);
            \node[node](m1)[above of=d2]{$*_3$}
            edge(d2);
            
            \node[node](p2)[above of=p1]{}
            edge(p1)
            edge(d2);
            \node[node](p3)[right of=p2]{}
            edge(d1)
            edge(m1);
            
            \node(y1)[above of=p2]{$y_1'$}
            edge(p2);
            \node(y2)[above of=p3]{$y_2'$}
            edge(p3);
          \end{scope}
        \end{tikzpicture}
      \end{column}
      \begin{column}{0.4\textwidth}
        \Large
        \begin{gather*}
          \begin{pmatrix}
            1 & 1 & 1\\
            0 & 1 & 3
          \end{pmatrix}\\
          \updownarrow\\
          \begin{pmatrix}
            1 & 0\\
            1 & 1\\
            1 & 3\\
          \end{pmatrix}
        \end{gather*}
      \end{column}
    \end{columns}
  \end{block}
\end{frame}


\section{A new way of looking}

\begin{frame}
  \frametitle{Category theory, a gentle introduction...}
  
  \begin{columns}[t]
    
    \begin{column}{0.3\textwidth}
      \begin{itemize}
      \item Category $\C$
      \item Objects $\ob(\C)$
      \item<2-> Arrows $\hom(\C)$, $\Hom(A,B)$
      \item<3-> Identities $\id_A$
      \item<4-> Composition $g\comp f$
      \end{itemize}
    \end{column}

    \begin{column}{0.7\textwidth}
      \begin{equation*}
        \xymatrix{
          A \only<2->{\ar@/^/[r] \ar@/_/[r]} \only<3->{\ar@(ul,dl)} & B \only<3->{\ar@(ur,dr)}\\
          &  & C \only<2->{\ar[ddr]^g} \only<3->{\ar@(ul,ur)} \\
          \\
          & D \only<2->{\ar[uur]^f} \only<3->{\ar@(dl,dr)} \only<4->{\ar[rr]^{g\comp f}}&   & E \only<3->{\ar@(dr,dl)}
        }
      \end{equation*}
      
      \vfill
    \end{column}
  \end{columns}

  \begin{block}<5>{Example : $\lcat{FMod}_R$}
    \begin{itemize}
    \item $\ob(\C) = R^n$ free $R$-modules,
    \item $\hom(\C) = $ linear applications.
    \end{itemize}
  \end{block}

\end{frame}


\begin{frame}
  \frametitle{Category theory, Functors}
  
  \begin{columns}
    \begin{column}{0.3\textwidth}
      Covariant functor $F:\C\ra\D$
    \end{column}

    \begin{column}{0.25\textwidth}
      \begin{equation*}
        \xymatrix{
          A \ar[r]^f \ar[dr]_{h} & B \ar[d]^g \\
          & C
        }
      \end{equation*}
    \end{column}
    \begin{column}{0.05\textwidth}
      $\mapsto$
    \end{column}
    \begin{column}{0.35\textwidth}
      \begin{equation*}
        \xymatrix{
          F(A) \ar[r]^{F(f)} \ar[dr]_{F(h)} & F(B) \ar[d]^{F(g)} \\
          & F(C)
        }    
      \end{equation*}
    \end{column}
  \end{columns}

  \begin{columns}
    \begin{column}{0.3\textwidth}
      Contravariant functor $F:\C\ra\D$
    \end{column}

    \begin{column}{0.25\textwidth}
      \begin{equation*}
        \xymatrix{
          A \ar[r]^f \ar[dr]_{h} & B \ar[d]^g \\
          & C
        }
      \end{equation*}
    \end{column}
    \begin{column}{0.05\textwidth}
      $\mapsto$
    \end{column}
    \begin{column}{0.35\textwidth}
      \begin{equation*}
        \xymatrix{
          F(A) & F(B) \ar[l]_{F(f)} \\
          & F(C) \ar[u]_{F(g)} \ar[ul]^{F(h)}
        }    
      \end{equation*}
    \end{column}
  \end{columns}

  \begin{block}{Equivalence, duality}
    \begin{itemize}
    \item Equivalence if $F:\C\ra\D$ and $G:\D\ra\C$ covariant
    \item Duality if $F:\C\ra\D$ and $G:\D\ra\C$ contravariant 
    \end{itemize}
    and $\quad F\comp G \simeq \Id_\D\quad$ and $\quad G\comp F \simeq \Id_\C$.
  \end{block}
  
\end{frame}


\begin{frame}[fragile]
  \frametitle{An example}

  \Large

  \begin{columns}

    \begin{column}{0.6\textwidth}
      \begin{center}
        \begin{minipage}{0.9\textwidth}
\begin{semiverbatim}
  for i = 1 to n-2 do
    a[i+1] = a[i] + a[i+1]
    a[i] = 0
  end for
\end{semiverbatim}
        \end{minipage}
      \end{center}
    \end{column}

    \begin{column}{0.4\textwidth}

      \begin{equation*}
        \begin{pmatrix}
          0 & \hdotsfor{3} & 0\\
          \vdots  &  &\vdots&& \vdots \\
          0 & \hdotsfor{3} & 0\\
          1 & \hdotsfor{3} & 1
        \end{pmatrix}
      \end{equation*}

    \end{column}
  \end{columns}
\end{frame}


\begin{frame}
  \frametitle{Computations}

  \begin{block}{Language, size}
    \begin{itemize}
    \item Set of \emph{instructions} \hfill $\L \subset \hom(\C)$,
    \item Size function \hfill $\size{.} : \ob(\C)\ra\N$.
    \end{itemize}
  \end{block}

  \begin{block}{Computation}
    \begin{center}
      Sequence $\quad C_1:b\ra c\quad$ of instructions.
    \end{center}
  \end{block}

  \[\xymatrix{
    & \ar[r]\ar@{}[dr]|{C_2} & \ar[dr] &
    & \ar[r]\ar@{}[dr]|{C_1} & \ar[dr] \\
    a \ar[ur]\ar@{-->}[rrr]_{f_2} &&& b
    \ar[ur]\ar@{-->}[rrr]_{f_1} &&& c
  }\]

  \begin{block}{Time and space cost}
    \begin{itemize}
    \item $\Time(C) = $ length of the computation,
    \item $\Space(C) = \max_{o\in C}\size{o}$.
    \end{itemize}
  \end{block}

\end{frame}


\begin{frame}
  \frametitle{The case $\lcat{FMod}_R$}
  [A.~Bostan, G.~Lecerf, E.~Schost 2003] :

  \begin{align*}
    +_1 : R^2 &\ra R^2         &   +_2 : R^2&\ra R^2       &  *_a : R&\ra R\\
    (p,q)&\mapsto(p+q,q)  &      (p,q)&\mapsto(p,p+q) &       p&\mapsto ap
  \end{align*}
  \begin{align*}
    \pi : R&\ra 0     &  \iota : 0&\ra R   \\
          p&\mapsto0  &          0&\mapsto0
  \end{align*}
  
  \begin{block}{}
    \begin{equation*}
      \label{eq:linlang}
      \L = \Bigl\{\Id_n\times\mathrm{op}\times\Id_m \Bigl\lvert
      n,m\in\N, \mathrm{op}\in\{+_1,+_2,*_a,\pi,\iota\;|\;a\in R\} \Bigr\}
      \text{.}
    \end{equation*}
  \end{block}

  \begin{block}{}
    \begin{center}
      $\size{R^n} = n$
    \end{center}
  \end{block}

\end{frame}


\begin{frame}[fragile]
  \frametitle{Our example}

  \Large

  \begin{columns}

    \begin{column}{0.6\textwidth}
      \begin{center}
        \begin{minipage}{\textwidth}
\begin{semiverbatim}
  \alert<2>{for i = 1 to n-2 do}
    a[i+1] = a[i] + a[i+1]
    a[i] = 0
  end for
\end{semiverbatim}
        \end{minipage}
      \end{center}
    \end{column}

    \begin{column}{0.4\textwidth}
      \begin{minipage}{\textwidth}
        \begin{flushright}
          $\phantom{ }$\\
          $\Id_{i}\times +_2 \times\Id_{n-2-i}$\\
          $\Id_i\times *_0 \times\Id_{n-1-i}$\\
        \end{flushright}
      \end{minipage}
    \end{column}
  \end{columns}

  \vfill

  \begin{center}
    \[\xymatrix@C=3.5em{
      R^n\ar[r]^{+_2\times{\Id_{n-2}}} & R^n\ar[r]^{*_0\times{\Id_{n-1}}} &
      R^{n} \ar@{.}[rr] && R^n
    }\]
  \end{center}
\end{frame}


\begin{frame}[fragile]
  \frametitle{Branchings}

  \Large

  \begin{center}
    \[\xymatrix{
             &        &                 & \ar@{.}[r] & \\
      \ar[r] & \ar[r] & \ar[ru] \ar[rd] \\
             &        &                 & \ar@{.}[r] & 
    }\]
  \end{center}

  \begin{center}
    \begin{minipage}{0.7\textwidth}
\begin{semiverbatim}
  \alert<2>{if \alt<1>{a = (0,...,0)}{n = 0} then}
    ...
  else
    ...
  endif
\end{semiverbatim}
    \end{minipage}
  \end{center}
  
\end{frame}


\begin{frame}
  \frametitle{Algorithm}

  \begin{block}{Parameter space}
    \begin{center}
      $\Par$ a recursively enumerable
    \end{center}
    \begin{center}
      For example, $\quad \Par=\N$
    \end{center}
  \end{block}

  \begin{block}{Algorithm}
    \begin{center}
      A function $\qquad A:\Par \ra \C_\ra \qquad$ ($\C_\ra =$ the computations)
    \end{center}
  \end{block}

  \begin{columns}

    \begin{column}{0.1\textwidth}
      \begin{align*}
        \uncover<2->{1}\\
        \uncover<3->{2}\\
        \uncover<4->{3}\\
        \uncover<4->{\vdots}
      \end{align*}
    \end{column}
    \begin{column}{0.9\textwidth}
      \begin{equation*}
        \xymatrix{
          \only<2->{\ar[r]} & \only<2->{\ar[rd]} & & \only<3->{\ar[r]}&\\
                            &                    & \only<2->{\ar[r]} \only<3->{\ar[ru]}&\\
          \only<3>{\ar[r]} \only<4->{\ar@{=>}[r]} & \only<3->{\ar[ur]}\only<4->{\ar[r]} & \only<4->{\ar[r]} & \only<4->{\ar[r]} &
        }
      \end{equation*}
    \end{column}
  \end{columns}
  
\end{frame}


\begin{frame}
  \frametitle{Complexity}

  \begin{block}{Time complexity}
    \begin{center}
      $A:\Par\ra\C_\ra\qquad$ induces a function
      $\qquad\Time_A:\Par\ra\N\qquad$ given by
      \[\Time_A(x) = \Time(A(x))\]
    \end{center}
  \end{block}

  \begin{block}{Space complexity}
    \begin{center}
      $A:\Par\ra\C_\ra\qquad$ induces a function
      $\qquad\Space_A:\Par\ra\N\qquad$ given by
      \[\Space_A(x) = \Space(A(x))\]
    \end{center}
  \end{block}
\end{frame}



\begin{frame}[fragile]
  \frametitle{Our example}

  \large

  \begin{center}
    \begin{minipage}{0.7\textwidth}
\begin{semiverbatim}
\only<1>{  for i = 1 to n-2 do
    a[i+1] = a[i] + a[i+1]
    a[i] = 0
  end for}
\only<2>{  a[1] = a[0] + a[1]
  a[0] = 0
  a[2] = a[1] + a[2]
  a[1] = 0
  ...
  a[n-1] = a[n-2] + a[n-1]
  a[n-2] = 0}
\end{semiverbatim}
    \end{minipage}
  \end{center}

  \vfill

  \begin{center}
    \[\uncover<2>{n \quad\mapsto\quad}\xymatrix@C=3.5em{
      R^n\ar[r]^{+_2\times{\Id_{n-2}}} & R^n\ar[r]^{*_0\times{\Id_{n-1}}} &
      R^{n} \ar@{.}[rr] && R^n
    }\]
  \end{center}
\end{frame}


\begin{frame}
  \frametitle{Tellegen's theorem}

  \begin{block}{T-functor}
    A functor $\qquad F:\C\ra\D \qquad$ is said to be a T-functor if
    $F(\L_\C) \subset \L_\D$.
  \end{block}

  \begin{block}{Tellegen's theorem}
    \begin{itemize}
    \item $F:\C\ra\D$ a T-functor
    \item $\Par$ a parameter space
    \item $A:\Par\ra\C_\ra$ an algorithm
    \end{itemize}

    $F\comp A$, noted $F(A)$ is an algorithm $\Par\ra\D_\ra$ such that
    \begin{itemize}
    \item $\Time_{F(A)} = \Time_A$,
    \item $\Space_{F(A)} \le B(\Space_A)$ if $B:\N\ra\N$ is an upper
      bound for $F$.
    \end{itemize}
  \end{block}
\end{frame}


\begin{frame}
  \frametitle{The case $\lcat{FMod}_R$}
  
  \begin{center}
    We know a (contravariant) functor $\quad
    T:\lcat{FMod}_R\ra\lcat{FMod}_R\quad$ given by matrix
    transposition.
  \end{center}

  \begin{block}{Tellegen's theorem for linear algebra}
    $T$ is a T-functor for the language $\L$ we gave before.
  \end{block}

  \begin{align*}
    T(+_1) = +_2 \qquad T(*_a) = *_a \qquad T(\pi) = \iota
  \end{align*}
\end{frame}



\section{Examples}

\begin{frame}[fragile]
  \frametitle{Our example}

  \begin{columns}
    \begin{column}{0.5\textwidth}
      \begin{center}
        \begin{minipage}{0.7\textwidth}
\begin{semiverbatim}
  a[1] = a[0] + a[1]
  a[0] = 0
  a[2] = a[1] + a[2]
  a[1] = 0
  ...
  a[n-1] = a[n-2] + a[n-1]
  a[n-2] = 0
\end{semiverbatim}
        \end{minipage}
      \end{center}
    \end{column}

    \begin{column}{0.5\textwidth}
      \begin{center}
        \begin{minipage}{0.7\textwidth}
\begin{semiverbatim}
  a[n-2] = 0
  a[n-2] = a[n-2] + a[n-1]
  ...
  a[1] = 0
  a[1] = a[1] + a[2]
  a[0] = 0
  a[0] = a[0] + a[1]
\end{semiverbatim}
        \end{minipage}
      \end{center}
    \end{column}
    \end{columns}
  
  \vfill
  \pause

  \begin{columns}

    \begin{column}{0.5\textwidth}
      \begin{center}
        \begin{minipage}{0.7\textwidth}
\begin{semiverbatim}
  for i = n-2 to 0 do
    a[i] = 0
    a[i] = a[i] + a[i+1]
  end for
\end{semiverbatim}
        \end{minipage}
      \end{center}
    \end{column}

    \begin{column}{0.5\textwidth}

      \begin{equation*}
        \begin{pmatrix}
          0 & \hdotsfor{3} & 0\\
          \vdots  &  &\vdots&& \vdots \\
          0 & \hdotsfor{3} & 0\\
          1 & \hdotsfor{3} & 1
        \end{pmatrix}
      \end{equation*}

    \end{column}
  \end{columns}
\end{frame}

\begin{frame}
  \frametitle{Other T-functors: Quantum Computing}

  \begin{block}{The QC category $\cat{Q}$}
    \begin{center}
    $\ob(\cat{Q}) = \{\left(\Complex^2\right)^{\otimes n}\;|\;n\in\N\}
    \qquad\qquad \hom(\cat{Q}) = U \text{ unitaries}$
    \end{center}
  \end{block}

  \begin{equation*}
    H = \frac{1}{\sqrt{2}}\begin{pmatrix}
      1 & -1 \\ -1 & 1
    \end{pmatrix},
    \qquad
    R = \begin{pmatrix}
      1 & 0 \\ 0 & e^{2\pi i }
    \end{pmatrix},
    \qquad
    CNOT = \begin{pmatrix}
      1 & 0 & 0 & 0\\
      0 & 1 & 0 & 0\\
      0 & 0 & 0 & 1\\
      0 & 0 & 1 & 0
    \end{pmatrix}
  \end{equation*}

  \begin{block}{Language, size}
    $\L = \Bigl\{\Id_2^{\otimes
      n}\otimes\mathrm{op}\otimes\Id_2^{\otimes m}\Bigl\lvert
    n,m\in\N, \mathrm{op}\in\{H,R,R^\ast,CNOT\} \Bigr\}$

    $\size{(\Complex^2)^{\otimes n}} = n$
  \end{block}
  
  \begin{block}{The functor}
    \begin{center}
      $\ast : U\mapsto U^\ast$
    \end{center}
  \end{block}
\end{frame}


\begin{frame}
  \frametitle{Other T-functors: Extension and restriction of scalars}

  \begin{block}{}
    $A,B$ two rings, $f:A\ra B$ a morphism,
    \begin{itemize}
    \item $E_f : M_A\mapsto M_A\otimes_AB$ maps $A$-modules to
      $B$-modules;
    \item $R_F : M_B\mapsto M_B$ maps $B$-modules to $A$-modules (by
      the law $am \equiv f(a)m$).
    \end{itemize}
  \end{block}

  \begin{block}{Extension}
    \begin{itemize}
    \item Every algorithm written for $\R$-modules works for
      $\Complex$-modules;
    \item Every algorithm written for $\Z$-modules works for any module;
    \item Every algorithm written for $\K[X]$-modules works for
      $K[X]/P(X)$ modules.
    \end{itemize}
  \end{block}

  \begin{block}{Restriction}
    \begin{itemize}
    \item Less important than extension;
    \item its \alert{adjoint}, not its inverse (extension of scalars
      may loose information).
    \end{itemize}
  \end{block}
\end{frame}

\begin{frame}
  \frametitle{Conclusions}

  \begin{block}{Towards theory}
    \begin{itemize}
    \item Is this new point of view more enlightening?
    \item Are there any other interesting T-functors?
    \end{itemize}
  \end{block}

  \begin{block}{Towards practice}
    \begin{itemize}
    \item Is it always possible to find the minimal parameter space?
    \item How much does the non-linear code (conditionals and loops)
      change when we transpose?
    \item Can we dream of a working AT tool?
    \end{itemize}
  \end{block}
\end{frame}

\end{document}




% Local Variables:
% mode:flyspell
% ispell-local-dictionary:"british"
% End:

% LocalWords:  Tellegen tellegenish
