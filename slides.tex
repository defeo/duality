\documentclass[10pt,draft]{beamer}

\usepackage[francais]{babel}
\usepackage[utf8]{inputenc}
\usepackage{amsmath,amssymb,amsthm,latexsym}
\usepackage{mathrsfs}
\usepackage[all]{xypic}

\newcommand{\cat}[1]{\mathscr{#1}}
\newcommand{\lcat}[1]{\mathbf{#1}}
\newcommand{\C}{\cat{C}}
\newcommand{\D}{\cat{D}}
\renewcommand{\L}{\cat{L}}
\newcommand{\comp}{\circ}
\newcommand{\size}[1]{\lVert#1\rVert}
\newcommand{\sizein}[1]{\size{#1}^\mathrm{in}}
\newcommand{\sizeout}[1]{\size{#1}_\mathrm{out}}
\DeclareMathOperator{\ob}{ob}
\DeclareMathOperator{\Hom}{Hom}
\DeclareMathOperator{\id}{id}
\DeclareMathOperator{\Id}{Id}
\newcommand{\N}{\mathbb{N}}
\newcommand{\Z}{\mathbb{Z}}
\newcommand{\Complex}{\mathbb{C}}
\newcommand{\ra}{\rightarrow}
\newcommand{\la}{\leftarrow}
\DeclareMathOperator{\Time}{t}
\DeclareMathOperator{\Space}{s}
\DeclareMathOperator{\coTime}{co-t}
\DeclareMathOperator{\coSpace}{co-s}
\DeclareMathOperator{\Par}{Par}
\newcommand{\computes}{\vdash}


\mode<presentation>{%
  \usetheme[]{Madrid}
  \usefonttheme[onlymath]{serif}
}


\title[Principe de transposition]{Principe de transposition
  \only<1>{et tours d'Artin-Schreier}}
\author{L.~De~Feo}
\institute[]{LIX, École Polytechnique}


\AtBeginSection[]
{
  \begin{frame}<beamer>
    \frametitle{Plan}
    \tableofcontents[currentsection]
  \end{frame}
}


\begin{document}

\begin{frame}<1,2>
  \titlepage
\end{frame}


\begin{frame}
  \frametitle{Le principe de Tellegen}

  \Large
  \begin{center}
    ``De tout \emph{algorithme linéaire} qui calcule une application
    linéaire on peut déduire an autre \emph{algorithme linéaire} qui
    calcule la transposée de l'application ayant \emph{à peu près} les
    mêmes complexités en espace et temps.''
  \end{center}

  \vfill
  \pause

  \begin{center}
    Qu'est-ce qu'il y a de tellement spécial dans la transposition ?
  \end{center}
\end{frame}


\begin{frame}
  \frametitle{Théorie des catégories, Généralités}
  
  \begin{columns}[t]
    
    \begin{column}{0.3\textwidth}
      \begin{itemize}
      \item Catégorie $\C$
      \item Objets $\ob(\C)$
      \item<2-> Flèches $\hom(\C)$, $\Hom(A,B)$
      \item<3-> Identités $\id_A$
      \item<4-> Composition $g\comp f$
      \end{itemize}
    \end{column}

    \begin{column}{0.7\textwidth}
      \begin{equation*}
        \xymatrix{
          A \only<2->{\ar@/^/[r] \ar@/_/[r]} \only<3->{\ar@(ul,dl)} & B \only<3->{\ar@(ur,dr)}\\
          &  & C \only<2->{\ar[ddr]^g} \only<3->{\ar@(ul,ur)} \\
          \\
          & D \only<2->{\ar[uur]^f} \only<3->{\ar@(dl,dr)} \only<4->{\ar[rr]^{g\comp f}}&   & E \only<3->{\ar@(dr,dl)}
        }
      \end{equation*}
      
      \vfill
    \end{column}
  \end{columns}

  \begin{block}<5>{Exemple : $\lcat{FMod}_R$}
    \begin{itemize}
    \item $\ob(\C) = R^n$ les $R$-modules libres,
    \item $\hom(\C) = $ les applications linéaires.
    \end{itemize}
  \end{block}

\end{frame}


\begin{frame}
  \frametitle{Théorie des catégories, Foncteurs}
  
  \begin{columns}
    \begin{column}{0.3\textwidth}
      Foncteur covariant $F:\C\ra\D$
    \end{column}

    \begin{column}{0.25\textwidth}
      \begin{equation*}
        \xymatrix{
          A \ar[r]^f \ar[dr]_{h} & B \ar[d]^g \\
          & C
        }
      \end{equation*}
    \end{column}
    \begin{column}{0.05\textwidth}
      $\mapsto$
    \end{column}
    \begin{column}{0.35\textwidth}
      \begin{equation*}
        \xymatrix{
          F(A) \ar[r]^{F(f)} \ar[dr]_{F(h)} & F(B) \ar[d]^{F(g)} \\
          & F(C)
        }    
      \end{equation*}
    \end{column}
  \end{columns}

  \begin{columns}
    \begin{column}{0.3\textwidth}
      Foncteur contravariant $F:\C\ra\D$
    \end{column}

    \begin{column}{0.25\textwidth}
      \begin{equation*}
        \xymatrix{
          A \ar[r]^f \ar[dr]_{h} & B \ar[d]^g \\
          & C
        }
      \end{equation*}
    \end{column}
    \begin{column}{0.05\textwidth}
      $\mapsto$
    \end{column}
    \begin{column}{0.35\textwidth}
      \begin{equation*}
        \xymatrix{
          F(A) & F(B) \ar[l]_{F(f)} \\
          & F(C) \ar[u]_{F(g)} \ar[ul]^{F(h)}
        }    
      \end{equation*}
    \end{column}
  \end{columns}

  \begin{block}{Équivalence, dualité}
    \begin{itemize}
    \item Équivalence si $F:\C\ra\D$ et $G:\D\ra\C$ covariants 
    \item Dualité si $F:\C\ra\D$ et $G:\D\ra\C$ contravariants 
    \end{itemize}
    et $\quad F\comp G \simeq \Id_\D\quad$ et $\quad G\comp F \simeq \Id_\C$.
  \end{block}
  
\end{frame}




\end{document}


% Local Variables:
% mode:flyspell
% ispell-local-dictionary:"francais"
% End:
