\documentclass[10pt]{beamer}

\usepackage[english]{babel}
\usepackage[utf8]{inputenc}
\usepackage{amsmath,amssymb,amsthm,latexsym}
\usepackage{mathrsfs}
\usepackage[all]{xypic}

\newcommand{\cat}[1]{\mathscr{#1}}
\newcommand{\lcat}[1]{\mathbf{#1}}
\newcommand{\C}{\cat{C}}
\newcommand{\D}{\cat{D}}
\renewcommand{\L}{\cat{L}}
\newcommand{\comp}{\circ}
\newcommand{\size}[1]{\lVert#1\rVert}
\newcommand{\sizein}[1]{\size{#1}^\mathrm{in}}
\newcommand{\sizeout}[1]{\size{#1}_\mathrm{out}}
\DeclareMathOperator{\ob}{ob}
\DeclareMathOperator{\Hom}{Hom}
\DeclareMathOperator{\id}{id}
\DeclareMathOperator{\Id}{Id}
\newcommand{\N}{\mathbb{N}}
\newcommand{\Z}{\mathbb{Z}}
\newcommand{\Complex}{\mathbb{C}}
\newcommand{\R}{\mathbb{R}}
\newcommand{\K}{\mathbb{K}}
\newcommand{\ra}{\rightarrow}
\newcommand{\la}{\leftarrow}
\DeclareMathOperator{\Time}{t}
\DeclareMathOperator{\Space}{s}
\DeclareMathOperator{\coTime}{co-t}
\DeclareMathOperator{\coSpace}{co-s}
\DeclareMathOperator{\Par}{Par}
\newcommand{\computes}{\vdash}


\mode<presentation>{%
  \usetheme[]{Madrid}
  \usefonttheme[onlymath]{serif}
}


\title[Dualities and Transposition Principle]{Dualities and
  Transposition Principle}
\author{L.~De~Feo}
\institute[LIX]{LIX, École Polytechnique, France}
\date[UWO, 11-21-2008]{SCL, University of Western Ontario\\November 21, 2008}

\AtBeginSection[]
{
  \begin{frame}<beamer>
    \frametitle{Plan}
    \tableofcontents[currentsection]
  \end{frame}
}


\begin{document}

\begin{frame}
  \titlepage
\end{frame}


\begin{frame}
  \frametitle{Tellegen's Principle}

  \Large
  \begin{center}
    ``From every \emph{linear algorithm} computing a
    linear application we can deduce another \emph{linear algorithm}
    computing the transpose application using \emph{about} the same
    space and time resources.''
  \end{center}

  \vfill
  \pause

  \begin{center}
    What's so special about transposition ?
  \end{center}
\end{frame}

\section{Background}

\begin{frame}
  \frametitle{History, motivations}
  
  \begin{block}{History}
    \begin{itemize}
    \item Originally discovered by \alert{Tellegen (1950)},
      \alert{Bordewijk (1956)} for \emph{electrical network theory}
      and by \alert{Kalman (1960)} for \emph{control theory};
    \item Graph-theoretic approach by \alert{Fettweis (1971)} for
      \emph{digital filters};
    \item \alert{Fiduccia (1972)}: transposition of \emph{bilinear
      algorithms};
    \item Special case of reverse mode in \emph{automatic differentiation}:
      \alert{Baur \& Strassen (1983)};
    \item In \emph{computer algebra}, popularized by \alert{Shoup},
      \alert{von zur Gathen}, \alert{Kaltofen},\dots
    \item \alert{[Bostan, Lecerf, Schost 2003]} improve algorithms for
      polynomial evaluation.
    \end{itemize}
  \end{block}
  
  \begin{block}{Motivations}
    \begin{itemize}
    \item Existence result in \emph{complexity theory};
    \item \emph{Code transformation} technique;
    \item Improve $M^T \Leftrightarrow$ Improve $M$;
    \item {\bf Divides by $2$ the number of algorithms yet to be discovered.}
    \end{itemize}
  \end{block}
\end{frame}


\begin{frame}
  \frametitle{Classical proofs}

  \begin{block}{Linear algebra}
    $M$ computed as a sequence of \emph{simple} linear applications $U_i$
    \begin{equation*}
      M(v) = U_1\circ U_2\circ\cdots\circ U_n(v)
      \qquad\Leftrightarrow\qquad
      M^T(v) = U_n^T\circ\cdots\circ U_2^T\circ U_1^T
    \end{equation*}
  \end{block}

  \begin{block}{Graph-theoretic approach}
    \begin{itemize}
    \item \emph{Compile} the algorithm in a DAG;
    \item reverse the arrows of the DAG.
    \end{itemize}
    \begin{center}
      \textbf{This works only for straight-line programs !}
    \end{center}
  \end{block}
\end{frame}


\begin{frame}
  \frametitle{Graph-theoretic approach (cont'd)}

  %% Well, SCL at UWO has a nice multifunctional blackboard (actually,
  %% a whiteboard) they use as a screen for their presentations. And I
  %% don't know how to draw a graph. I think I'll draw this on the
  %% blackboard!
\end{frame}


\section{A new way of looking}

\begin{frame}
  \frametitle{Category theory, a gentle introduction...}
  
  \begin{columns}[t]
    
    \begin{column}{0.3\textwidth}
      \begin{itemize}
      \item Category $\C$
      \item Objects $\ob(\C)$
      \item<2-> Arrows $\hom(\C)$, $\Hom(A,B)$
      \item<3-> Identities $\id_A$
      \item<4-> Composition $g\comp f$
      \end{itemize}
    \end{column}

    \begin{column}{0.7\textwidth}
      \begin{equation*}
        \xymatrix{
          A \only<2->{\ar@/^/[r] \ar@/_/[r]} \only<3->{\ar@(ul,dl)} & B \only<3->{\ar@(ur,dr)}\\
          &  & C \only<2->{\ar[ddr]^g} \only<3->{\ar@(ul,ur)} \\
          \\
          & D \only<2->{\ar[uur]^f} \only<3->{\ar@(dl,dr)} \only<4->{\ar[rr]^{g\comp f}}&   & E \only<3->{\ar@(dr,dl)}
        }
      \end{equation*}
      
      \vfill
    \end{column}
  \end{columns}

  \begin{block}<5>{Example : $\lcat{FMod}_R$}
    \begin{itemize}
    \item $\ob(\C) = R^n$ free $R$-modules,
    \item $\hom(\C) = $ linear applications.
    \end{itemize}
  \end{block}

\end{frame}


\begin{frame}
  \frametitle{Category theory, Functors}
  
  \begin{columns}
    \begin{column}{0.3\textwidth}
      Covariant functor $F:\C\ra\D$
    \end{column}

    \begin{column}{0.25\textwidth}
      \begin{equation*}
        \xymatrix{
          A \ar[r]^f \ar[dr]_{h} & B \ar[d]^g \\
          & C
        }
      \end{equation*}
    \end{column}
    \begin{column}{0.05\textwidth}
      $\mapsto$
    \end{column}
    \begin{column}{0.35\textwidth}
      \begin{equation*}
        \xymatrix{
          F(A) \ar[r]^{F(f)} \ar[dr]_{F(h)} & F(B) \ar[d]^{F(g)} \\
          & F(C)
        }    
      \end{equation*}
    \end{column}
  \end{columns}

  \begin{columns}
    \begin{column}{0.3\textwidth}
      Contravariant functor $F:\C\ra\D$
    \end{column}

    \begin{column}{0.25\textwidth}
      \begin{equation*}
        \xymatrix{
          A \ar[r]^f \ar[dr]_{h} & B \ar[d]^g \\
          & C
        }
      \end{equation*}
    \end{column}
    \begin{column}{0.05\textwidth}
      $\mapsto$
    \end{column}
    \begin{column}{0.35\textwidth}
      \begin{equation*}
        \xymatrix{
          F(A) & F(B) \ar[l]_{F(f)} \\
          & F(C) \ar[u]_{F(g)} \ar[ul]^{F(h)}
        }    
      \end{equation*}
    \end{column}
  \end{columns}

  \begin{block}{Equivalence, duality}
    \begin{itemize}
    \item Equivalence if $F:\C\ra\D$ and $G:\D\ra\C$ covariant
    \item Duality if $F:\C\ra\D$ and $G:\D\ra\C$ contravariant 
    \end{itemize}
    and $\quad F\comp G \simeq \Id_\D\quad$ and $\quad G\comp F \simeq \Id_\C$.
  \end{block}
  
\end{frame}


\begin{frame}
  \frametitle{Tellegen's principle}

  \Large
  \begin{center}
    ``From every \alert{\emph{linear algorithm}} computing a linear
    application we can deduce another \alert{\emph{linear algorithm}}
    computing the transpose application using \emph{about} the same
    space and time resources.''
  \end{center}
\end{frame}


\begin{frame}[fragile]
  \frametitle{An example}

  \begin{columns}

    \begin{column}{0.5\textwidth}
      \begin{center}
        \begin{minipage}{0.7\textwidth}
\begin{semiverbatim}
  for i = 1 to n-2 do
    a[i+1] = a[i] + a[i+1]
    a[i] = 0
  end for
\end{semiverbatim}
        \end{minipage}
      \end{center}
    \end{column}

    \begin{column}{0.5\textwidth}

      \begin{equation*}
        \begin{pmatrix}
          0 & \hdotsfor{3} & 0\\
          \vdots  &  &\vdots&& \vdots \\
          0 & \hdotsfor{3} & 0\\
          1 & \hdotsfor{3} & 1
        \end{pmatrix}
      \end{equation*}

    \end{column}
  \end{columns}
\end{frame}


\begin{frame}
  \frametitle{Computations}

  \begin{block}{Language, size}
    \begin{itemize}
    \item Set of \emph{instructions} \hfill $\L \subset \hom(\C)$,
    \item Size function \hfill $\size{.} : \ob(\C)\ra\N$.
    \end{itemize}
  \end{block}

  \begin{block}{Computation}
    \begin{center}
      Sequence $\quad C_1:b\ra c\quad$ of instructions.
    \end{center}
  \end{block}

  \[\xymatrix{
    & \ar[r]\ar@{}[dr]|{C_2} & \ar[dr] &
    & \ar[r]\ar@{}[dr]|{C_1} & \ar[dr] \\
    a \ar[ur]\ar@{-->}[rrr]_{f_2} &&& b
    \ar[ur]\ar@{-->}[rrr]_{f_1} &&& c
  }\]

  \begin{block}{Time and space cost}
    \begin{itemize}
    \item $\Time(C) = $ length of the computation,
    \item $\Space(C) = \max_{o\in C}\size{o}$.
    \end{itemize}
  \end{block}

\end{frame}


\begin{frame}
  \frametitle{The case $\lcat{FMod}_R$}
  [A.~Bostan, G.~Lecerf, E.~Schost 2003] :

  \begin{align*}
    +_1 : R^2 &\ra R^2         &   +_2 : R^2&\ra R^2       &  *_a : R&\ra R\\
    (p,q)&\mapsto(p+q,q)  &      (p,q)&\mapsto(p,p+q) &       p&\mapsto ap
  \end{align*}
  \begin{align*}
    \pi : R&\ra 0     &  \iota : 0&\ra R   \\
          p&\mapsto0  &          0&\mapsto0
  \end{align*}
  
  \begin{block}{}
    \begin{equation*}
      \label{eq:linlang}
      \L = \Bigl\{\Id_n\times\mathrm{op}\times\Id_m \Bigl\lvert
      n,m\in\N, \mathrm{op}\in\{+_1,+_2,*_a,\pi,\iota\;|\;a\in R\} \Bigr\}
      \text{.}
    \end{equation*}
  \end{block}

  \begin{block}{}
    \begin{center}
      $\size{R^n} = n$
    \end{center}
  \end{block}

\end{frame}


\begin{frame}[fragile]
  \frametitle{Our example}

  \begin{columns}

    \begin{column}{0.5\textwidth}
      \begin{center}
        \begin{minipage}{0.7\textwidth}
\begin{semiverbatim}
  \alert<2>{for i = 1 to n-2 do}
    a[i+1] = a[i] + a[i+1]
    a[i] = 0
  end for
\end{semiverbatim}
        \end{minipage}
      \end{center}
    \end{column}

    \begin{column}{0.5\textwidth}
      \begin{minipage}{0.7\textwidth}
        \begin{center}
          $\phantom{ }$\\
          $\Id_{i}\times +_2 \times\Id_{n-2-i}$\\
          $\Id_i\times *_0 \times\Id_{n-1-i}$\\
        \end{center}
      \end{minipage}
    \end{column}
  \end{columns}

  \vfill

  \begin{center}
    \[\xymatrix{
      R^n\ar[r]^{+_2\times{\Id_{n-2}}} & R^n\ar[r]^{\*_0\times{\Id_{n-1}}} &
      R^{n} \ar@{.}[rr] && R^n
    }\]
  \end{center}
\end{frame}


\begin{frame}[fragile]
  \frametitle{Branchings}

  \begin{center}
    \[\xymatrix{
             &        &                 & \ar@{.}[r] & \\
      \ar[r] & \ar[r] & \ar[ru] \ar[rd] \\
             &        &                 & \ar@{.}[r] & 
    }\]
  \end{center}

  \begin{center}
    \begin{minipage}{0.7\textwidth}
\begin{semiverbatim}
  \alert<2>{if \alt<1>{a = (0,...,0)}{n = 0} then}
    ...
  else
    ...
  endif
\end{semiverbatim}
    \end{minipage}
  \end{center}
  
\end{frame}


\begin{frame}
  \frametitle{Algorithm}

  \begin{block}{Parameter space}
    \begin{center}
      $\Par$ a recursively enumerable
    \end{center}
    \begin{center}
      For example, $\quad \Par=\N$
    \end{center}
  \end{block}

  \begin{block}{Algorithm}
    \begin{center}
      A function $\qquad A:\Par \ra \C_\ra \qquad$ ($\C_\ra =$ the computations)
    \end{center}
  \end{block}

  \begin{columns}

    \begin{column}{0.1\textwidth}
      \begin{align*}
        \uncover<2->{1}\\
        \uncover<3->{2}\\
        \uncover<4->{3}\\
        \uncover<4->{\vdots}
      \end{align*}
    \end{column}
    \begin{column}{0.9\textwidth}
      \begin{equation*}
        \xymatrix{
          \only<2->{\ar[r]} & \only<2->{\ar[rd]} & & \only<3->{\ar[r]}&\\
                            &                    & \only<2->{\ar[r]} \only<3->{\ar[ru]}&\\
          \only<3>{\ar[r]} \only<4->{\ar@{=>}[r]} & \only<3->{\ar[ur]}\only<4->{\ar[r]} & \only<4->{\ar[r]} & \only<4->{\ar[r]} &
        }
      \end{equation*}
    \end{column}
  \end{columns}
  
\end{frame}


\begin{frame}
  \frametitle{Complexity}

  \begin{block}{Time complexity}
    \begin{center}
      $A:\Par\ra\C_\ra\qquad$ induces a function
      $\qquad\Time_A:\Par\ra\N\qquad$ given by
      \[\Time_A(x) = \Time(A(x))\]
    \end{center}
  \end{block}

  \begin{block}{Space complexity}
    \begin{center}
      $A:\Par\ra\C_\ra\qquad$ induces a function
      $\qquad\Space_A:\Par\ra\N\qquad$ given by
      \[\Space_A(x) = \Space(A(x))\]
    \end{center}
  \end{block}
\end{frame}



\begin{frame}[fragile]
  \frametitle{Our example}

  \begin{center}
    \begin{minipage}{0.7\textwidth}
\begin{semiverbatim}
\only<1>{  for i = 1 to n-2 do
    a[i+1] = a[i] + a[i+1]
    a[i] = 0
  end for}
\only<2>{  a[1] = a[0] + a[1]
  a[0] = 0
  a[2] = a[1] + a[2]
  a[1] = 0
  ...
  a[n-1] = a[n-2] + a[n-1]
  a[n-2] = 0}
\end{semiverbatim}
    \end{minipage}
  \end{center}

  \vfill

  \begin{center}
    \[\uncover<2>{n \quad\mapsto\quad}\xymatrix{
      R^n\ar[r]^{+_2\times{\Id_{n-2}}} & R^n\ar[r]^{\*_0\times{\Id_{n-1}}} &
      R^{n} \ar@{.}[rr] && R^n
    }\]
  \end{center}
\end{frame}


\begin{frame}
  \frametitle{Tellegen's theorem}

  \begin{block}{Tellegen\only<2>{\alert{\emph{ish}}} functor}
    A functor $\qquad F:\C\ra\D \qquad$ is said to be
    \only<1>{Tellegen}\only<2>{tellegenish} if $F(\L_\C) \subset
    \L_\D$.
  \end{block}

  \begin{block}{Tellegen's theorem}
    \begin{itemize}
    \item $F:\C\ra\D$ a \only<1>{Tellegen}\only<2>{tellegenish}
      functor
    \item $\Par$ a parameter space
    \item $A:\Par\ra\C_\ra$ an algorithm
    \end{itemize}

    $F\comp A$, noted $F(A)$ is an algorithm $\Par\ra\D_\ra$ such that
    \begin{itemize}
    \item $\Time_{F(A)} = \Time_A$,
    \item $\Space_{F(A)} = B(\Space_A)$ if $B:\N\ra\N$ is an upper
      bound for $F$.
    \end{itemize}
  \end{block}
\end{frame}


\begin{frame}
  \frametitle{The case $\lcat{FMod}_R$}
  
  \begin{center}
    We know a (contravariant) functor $\quad
    T:\lcat{FMod}_R\ra\lcat{FMod}_R\quad$ given by matrix
    transposition.
  \end{center}

  \begin{block}{Tellegen's theorem for linear algebra}
    $T$ is a tellegenish functor for the language $\L$ we gave before.
  \end{block}

  \begin{align*}
    T(+_1) = +_2 \qquad T(*_a) = *_a \qquad T(\pi) = \iota
  \end{align*}
\end{frame}



\section{Examples}

\begin{frame}[fragile]
  \frametitle{Our example}

  \begin{columns}
    \begin{column}{0.5\textwidth}
      \begin{center}
        \begin{minipage}{0.7\textwidth}
\begin{semiverbatim}
  a[1] = a[0] + a[1]
  a[0] = 0
  a[2] = a[1] + a[2]
  a[1] = 0
  ...
  a[n-1] = a[n-2] + a[n-1]
  a[n-2] = 0
\end{semiverbatim}
        \end{minipage}
      \end{center}
    \end{column}

    \begin{column}{0.5\textwidth}
      \begin{center}
        \begin{minipage}{0.7\textwidth}
\begin{semiverbatim}
  a[n-2] = 0
  a[n-2] = a[n-2] + a[n-1]
  ...
  a[1] = 0
  a[1] = a[1] + a[2]
  a[0] = 0
  a[0] = a[0] + a[1]
\end{semiverbatim}
        \end{minipage}
      \end{center}
    \end{column}
    \end{columns}
  
  \vfill
  \pause

  \begin{columns}

    \begin{column}{0.5\textwidth}
      \begin{center}
        \begin{minipage}{0.7\textwidth}
\begin{semiverbatim}
  for i = n-2 to 0 do
    a[i] = 0
    a[i] = a[i] + a[i+1]
  end for
\end{semiverbatim}
        \end{minipage}
      \end{center}
    \end{column}

    \begin{column}{0.5\textwidth}

      \begin{equation*}
        \begin{pmatrix}
          0 & \hdotsfor{3} & 0\\
          \vdots  &  &\vdots&& \vdots \\
          0 & \hdotsfor{3} & 0\\
          1 & \hdotsfor{3} & 1
        \end{pmatrix}
      \end{equation*}

    \end{column}
  \end{columns}
\end{frame}

\begin{frame}
  \frametitle{Quantum Computing}

  \begin{block}{The QC category $\cat{Q}$}
    \begin{center}
    $\ob(\cat{Q}) = \{\left(\Complex^2\right)^{\otimes n}\;|\;n\in\N\}
    \qquad\qquad \hom(\cat{Q}) = U \text{ unitaries
      ($U^\ast=U^{-1}$)}$
    \end{center}
  \end{block}

  \begin{equation*}
    H = \frac{1}{\sqrt{2}}\begin{pmatrix}
      1 & -1 \\ -1 & 1
    \end{pmatrix},
    \qquad
    R = \begin{pmatrix}
      1 & 0 \\ 0 & e^{2\pi i }
    \end{pmatrix},
    \qquad
    CNOT = \begin{pmatrix}
      1 & 0 & 0 & 0\\
      0 & 1 & 0 & 0\\
      0 & 0 & 0 & 1\\
      0 & 0 & 1 & 0
    \end{pmatrix}
  \end{equation*}

  \begin{block}{Language, size}
    \begin{center}
      $\L = \Bigl\{\Id_2^{\otimes
        n}\otimes\mathrm{op}\otimes\Id_2^{\otimes m}\Bigl\lvert
      n,m\in\N, \mathrm{op}\in\{H,R,R^\ast,CNOT\} \Bigr\} \qquad
      \size{(\Complex^2)^{\otimes n}} = n$
    \end{center}
  \end{block}
  
  \begin{block}{The functor}
    \begin{center}
      $\ast : U\mapsto U^\ast$
    \end{center}
  \end{block}
\end{frame}


\begin{frame}
  \frametitle{Extension and restriction of scalars}

  \begin{block}{}
    $A,B$ two rings, $f:A\ra B$ a morphism,
    \begin{itemize}
    \item $E_f : M_A\mapsto M_A\otimes_AB$ maps $A$-modules to
      $B$-modules;
    \item $R_F : M_B\mapsto M_B$ maps $B$-modules to $A$-modules (by
      the law $am \equiv f(a)m$).
    \end{itemize}
  \end{block}

  \begin{block}{Extension}
    \begin{itemize}
    \item Every algorithm written for $\R$-modules works for
      $\Complex$-modules;
    \item Every algorithm written for $\Z$-modules works for any module;
    \item Every algorithm written for $\K[X]$-modules works for
      $K[X]/P(X)$ modules.
    \end{itemize}
  \end{block}

  \begin{block}{Restriction}
    \begin{itemize}
    \item Less important than extension;
    \item its \alert{adjoint}, not its inverse (extension of scalars
      may loose information).
    \end{itemize}
  \end{block}
\end{frame}

\begin{frame}
  \frametitle{Caveat : automatic differentiation}
  
  \begin{block}{Making AD tellegenish}
    \begin{itemize}
    \item $\R$ the only object,
    \item arrows : all the analytic functions,
    \item \textbf{What functor ?}
    \end{itemize}
  \end{block}

  \begin{block}{}
    \begin{center}
      One might be tempted to define the category whose arrows are the
      first derivatives of functions. But
      \[ f'\circ g' \ne (f\circ g)' \quad\text{ !!!}\]
      The composition is not even definable in this case !
    \end{center}
  \end{block}

  \begin{block}{}
    \begin{itemize}
    \item We can define an artificial category that lets us treat AD,
    \item but composition is \alert{not for free} !
    \end{itemize}
  \end{block}
\end{frame}


\section{Tellegen's principle into practice}

\begin{frame}[fragile]
  \tiny
  \frametitle{Automatic transposition of code}

  \begin{columns}
    \begin{column}{0.5\textwidth}
      \begin{center}
        \begin{minipage}{\textwidth}
\begin{verbatim}
void reduc_doit(GF2X& A0, GF2X& A1, const GF2X& A,
	long init, long d, bool plusone){
  if (d <= 2){
    A0 = GF2X(0, coeff(A,init));
    A1 = GF2X(0, coeff(A,init+1));
    return;
  }
   
  long dp = d/2;
  GF2X A10, A11;

  reduc_doit(A0, A1, A, init, dp, plusone);
  reduc_doit(A10, A11, A, init+dp, dp, plusone);
 
  ShiftAdd(A0, A11, 1);
  if (plusone) A0 += A11;
  A1 += A10 + A11;

  long i = 1;
  bool even = true;
  while (2*i != d){
    ShiftAdd(A0, A10, i);
    ShiftAdd(A1, A11, i);
    i = 2*i;
    even = !even;
  }
  
  if (plusone && !even) {
    A0 += A10;
    A1 += A11;
  }
}
\end{verbatim}
        \end{minipage}
      \end{center}
    \end{column}

    \begin{column}{0.5\textwidth}
      \begin{center}
        \begin{minipage}{\textwidth}
\begin{verbatim}
void treduc_doit(GF2X& A, const GF2X& A0, const GF2X& A1, long d,
	bool plusone){
  if (d <= 2){
    SetCoeff(A, 0, coeff(A0, 0));
    SetCoeff(A, 1, coeff(A1, 0));
    return;
  }
   
  long dp = d/2;
  long hdp = dp/2;

  GF2X A00, A01, A10, A11;
  A00 = trunc(A0, hdp);
  A01 = trunc(A1, hdp);

  A10 = A01;
  if (plusone) A11 = A00;
  else A11 = 0;
  A11 += A01 + RightShift(trunc(A0, hdp+1), 1);
  long i = 1;
  bool even = true;
  while (2*i != d){
    A10 += RightShift(trunc(A0, hdp+i), i);
    A11 += RightShift(trunc(A1, hdp+i), i);
    i = 2*i;
    even = !even;
  }
  
  if (plusone && !even) {
    A10 += trunc(A0, hdp);
    A11 += trunc(A1, hdp);
  }
  
  GF2X B0, B1;
  treduc_doit(B0, A00, A01, dp, plusone);
  treduc_doit(B1, A10, A11, dp, plusone);
  A = B0 + LeftShift(B1,dp);
}
\end{verbatim}
        \end{minipage}
      \end{center}
    \end{column}
    \end{columns}
\end{frame}


\begin{frame}
  \frametitle{Automatic transposition of code}

  \begin{center}
    \Large
    I want a compiler that automatically transposes my code !
  \end{center}

  \begin{block}{The problem}
    \begin{itemize}
    \item Fix two computational categories and a tellegenish functor $F$,
    \item the languages have to be \alert{reasonable},
    \item composition has to be \alert{trivial},
    \item deciding the image of an instruction by $F$ must be
      \alert{feasible}.
    \end{itemize}
  \end{block}

  \begin{block}{Straight line programs}
    \begin{itemize}
    \item Easy.
    \item Read the program upside-down or bottom-up (depending if the
      $F$ is covariant or contravariant),
    \item substitute each instruction with its dual.
    \end{itemize}
  \end{block}

\end{frame}


\begin{frame}[fragile]
  \frametitle{Subroutines}
  
  \begin{block}{Subroutines}
    \begin{columns}
      \begin{column}{0.33\textwidth}
        \begin{center}
          \begin{minipage}{0.7\textwidth}
            \footnotesize
\begin{verbatim}
f(x, y, z) {
  ...
  a = g(x, y);
  ...
}
\end{verbatim}    
          \end{minipage}
        \end{center}
      \end{column}
      \begin{column}{0.33\textwidth}
        \[\xymatrix{
          & \ar@{-->}[r]_g & \ar[dr] & \\
          \ar[ur]\ar@{-->}[rrr]_f &&& 
        }\]
      \end{column}
      \begin{column}{0.01\textwidth}
        $\Rightarrow$
      \end{column}
      \begin{column}{0.33\textwidth}
        \[\xymatrix{
          &\ar[dl] & \ar@{-->}[l]^{g^T} & \\
          &&& \ar[ul]\ar@{-->}[lll]^{f^T}
        }\]
        \end{column}
    \end{columns}
  \end{block}


  \begin{block}{Recursion}
    \[\xymatrix{
      &&& \ar[dr]\\
      & & \ar[ur]\ar@{-->}[rr]_{f_0} && \ar[dr] \\
      & \ar[ur] \ar@{-->}[rrrr]_{f_1} &&&& \ar[dr] \\
      \ar[ur] \ar@{-->}[rrrrrr]_{f_2} &&&&&&
    }\]
  \end{block}
\end{frame}


\begin{frame}[fragile]
  \frametitle{Conditionals, loops}
  
  \begin{block}{Conditionals}
    \begin{columns}
      \begin{column}{0.3\textwidth}
        \begin{center}
          \begin{minipage}{0.7\textwidth}
            \footnotesize
\begin{verbatim}
if n > 0 then
  ...
else
  ...
endif
\end{verbatim}    
          \end{minipage}
        \end{center}
      \end{column}
      \begin{column}{0.7\textwidth}
        \begin{itemize}
          \item \verb|n| \alert{must be} in the parameter space,
          \item the conditional is left unchanged.
        \end{itemize}
      \end{column}
    \end{columns}
  \end{block}

  \begin{block}{Loops}
    \begin{columns}
      \begin{column}{0.3\textwidth}
        \begin{center}
          \begin{minipage}{0.7\textwidth}
            \footnotesize
\begin{verbatim}
for i = 0 to n do
  ...
end for
\end{verbatim}    
          \end{minipage}
        \end{center}
      \end{column}
      \begin{column}{0.7\textwidth}
        \begin{itemize}
          \item \verb|n| \alert{must be} in the parameter space,
          \item the loop is turned upside down (from \verb|n| to \verb|0|).
          \item It also works for nested loops.
        \end{itemize}
      \end{column}
    \end{columns}    
  \end{block}

  \begin{center}
    \textbf{Are there any more complicated patterns ?}
  \end{center}
\end{frame}

\begin{frame}
  \frametitle{Who's in the parameter space?}
  
  \begin{block}{The minimal parameter space}
    \begin{itemize}
    \item Some variables \textbf{must} be in the parameter space.
    \item How do we find them ?
    \end{itemize}
  \end{block}

  \begin{block}{The maximal parameter space}
    \begin{itemize}
    \item Any variable \textbf{can} be in the parameter space.
    \item Any decision problem can expressed in this category:
      \[\xymatrix{ Y \ar@(dl,ul)^{\id_Y} & N \ar@(ur,dr)^{\id_N} }\]
    \item Even when we fix $\lcat{FMod}_{\Z}$,
    \item let our code compute a function $f :\{0,1\}^\ast\ra\{0,1\}^\ast$,
    \item Consider $A:\{0,1\}^\ast\ra\lcat{FMod}_{\Z}$ such that $A(x) =
      \ast_{f(x)}$,
    \item all the variables are in the parameter space.
    \end{itemize}
  \end{block}
\end{frame}


\begin{frame}[fragile]
  \frametitle{Who's in the parameter space?}
  
  \begin{block}{The case of multiplication}
    \begin{columns}
      \begin{column}{0.3\textwidth}
        \begin{center}
          \begin{minipage}{0.7\textwidth}
            \footnotesize
\begin{verbatim}
Mult(x, y) {
  return x * y;
}
\end{verbatim}    
          \end{minipage}
        \end{center}
      \end{column}
      \begin{column}{0.7\textwidth}
        \begin{itemize}
        \item Multiplication \alert{is not linear} (it is bilinear),
        \item But \emph{Transposed multiplication} (aka \emph{Middle
          product}) is a very important operation :
        \item fix \verb|x|, then $\verb|Mult|_{\verb|x|}$ \alert{is linear}.
        \end{itemize}
      \end{column}
    \end{columns}    
  \end{block}

  \begin{block}{The solution}
    \begin{center}
      \Large Put \verb|x| in the parameter space !
    \end{center}
  \end{block}

  \begin{center}
    \textbf{How do we automatically find it ?}
  \end{center}
\end{frame}

\begin{frame}
  \frametitle{Conclusions}

  \begin{block}{Towards theory}
    \begin{itemize}
    \item Is this new point of view more enlightening?
    \item Are there any other interesting tellegenish functors?
    \end{itemize}
  \end{block}

  \begin{block}{Towards practice}
    \begin{itemize}
    \item Is it always possible to find the minimal parameter space?
    \item How much does the code (conditionals and loops) change when
      we transpose?
    \item Can we dream of a working AT tool?
    \end{itemize}
  \end{block}
\end{frame}

\end{document}




% Local Variables:
% mode:flyspell
% ispell-local-dictionary:"british"
% End:

% LocalWords:  Tellegen tellegenish
