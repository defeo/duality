\documentclass[10pt]{beamer}

\usepackage[francais]{babel}
\usepackage[utf8]{inputenc}
\usepackage{amsmath,amssymb,amsthm,latexsym}
\usepackage{mathrsfs}
\usepackage[all]{xypic}

\newcommand{\cat}[1]{\mathscr{#1}}
\newcommand{\lcat}[1]{\mathbf{#1}}
\newcommand{\C}{\cat{C}}
\newcommand{\D}{\cat{D}}
\renewcommand{\L}{\cat{L}}
\newcommand{\comp}{\circ}
\newcommand{\size}[1]{\lVert#1\rVert}
\newcommand{\sizein}[1]{\size{#1}^\mathrm{in}}
\newcommand{\sizeout}[1]{\size{#1}_\mathrm{out}}
\DeclareMathOperator{\ob}{ob}
\DeclareMathOperator{\Hom}{Hom}
\DeclareMathOperator{\id}{id}
\DeclareMathOperator{\Id}{Id}
\newcommand{\N}{\mathbb{N}}
\newcommand{\Z}{\mathbb{Z}}
\newcommand{\Complex}{\mathbb{C}}
\newcommand{\ra}{\rightarrow}
\newcommand{\la}{\leftarrow}
\DeclareMathOperator{\Time}{t}
\DeclareMathOperator{\Space}{s}
\DeclareMathOperator{\coTime}{co-t}
\DeclareMathOperator{\coSpace}{co-s}
\DeclareMathOperator{\Par}{Par}
\newcommand{\computes}{\vdash}


\mode<presentation>{%
  \usetheme[]{Madrid}
  \usefonttheme[onlymath]{serif}
}


\title[Principe de transposition]{Principe de transposition
  \only<1>{et tours d'Artin-Schreier}}
\author{L.~De~Feo}
\institute[LIX]{LIX, École Polytechnique}
\date[JNCF 2008]{JNCF 2008\\ Luminy, 20 octobre 2008}

%% \AtBeginSection[]
%% {
%%   \begin{frame}<beamer>
%%     \frametitle{Plan}
%%     \tableofcontents[currentsection]
%%   \end{frame}
%% }


\begin{document}

\begin{frame}<1,2>
  \titlepage
\end{frame}


\begin{frame}
  \frametitle{Le principe de Tellegen}

  \Large
  \begin{center}
    ``De tout \emph{algorithme linéaire} qui calcule une application
    linéaire on peut déduire an autre \emph{algorithme linéaire} qui
    calcule la transposée de l'application ayant \emph{à peu près} les
    mêmes complexités en espace et temps.''
  \end{center}

  \vfill
  \pause

  \begin{center}
    Qu'est-ce qu'il y a de tellement spécial dans la transposition ?
  \end{center}
\end{frame}


\begin{frame}
  \frametitle{Théorie des catégories, Généralités}
  
  \begin{columns}[t]
    
    \begin{column}{0.3\textwidth}
      \begin{itemize}
      \item Catégorie $\C$
      \item Objets $\ob(\C)$
      \item<2-> Flèches $\hom(\C)$, $\Hom(A,B)$
      \item<3-> Identités $\id_A$
      \item<4-> Composition $g\comp f$
      \end{itemize}
    \end{column}

    \begin{column}{0.7\textwidth}
      \begin{equation*}
        \xymatrix{
          A \only<2->{\ar@/^/[r] \ar@/_/[r]} \only<3->{\ar@(ul,dl)} & B \only<3->{\ar@(ur,dr)}\\
          &  & C \only<2->{\ar[ddr]^g} \only<3->{\ar@(ul,ur)} \\
          \\
          & D \only<2->{\ar[uur]^f} \only<3->{\ar@(dl,dr)} \only<4->{\ar[rr]^{g\comp f}}&   & E \only<3->{\ar@(dr,dl)}
        }
      \end{equation*}
      
      \vfill
    \end{column}
  \end{columns}

  \begin{block}<5>{Exemple : $\lcat{FMod}_R$}
    \begin{itemize}
    \item $\ob(\C) = R^n$ les $R$-modules libres,
    \item $\hom(\C) = $ les applications linéaires.
    \end{itemize}
  \end{block}

\end{frame}


\begin{frame}
  \frametitle{Théorie des catégories, Foncteurs}
  
  \begin{columns}
    \begin{column}{0.3\textwidth}
      Foncteur covariant $F:\C\ra\D$
    \end{column}

    \begin{column}{0.25\textwidth}
      \begin{equation*}
        \xymatrix{
          A \ar[r]^f \ar[dr]_{h} & B \ar[d]^g \\
          & C
        }
      \end{equation*}
    \end{column}
    \begin{column}{0.05\textwidth}
      $\mapsto$
    \end{column}
    \begin{column}{0.35\textwidth}
      \begin{equation*}
        \xymatrix{
          F(A) \ar[r]^{F(f)} \ar[dr]_{F(h)} & F(B) \ar[d]^{F(g)} \\
          & F(C)
        }    
      \end{equation*}
    \end{column}
  \end{columns}

  \begin{columns}
    \begin{column}{0.3\textwidth}
      Foncteur contravariant $F:\C\ra\D$
    \end{column}

    \begin{column}{0.25\textwidth}
      \begin{equation*}
        \xymatrix{
          A \ar[r]^f \ar[dr]_{h} & B \ar[d]^g \\
          & C
        }
      \end{equation*}
    \end{column}
    \begin{column}{0.05\textwidth}
      $\mapsto$
    \end{column}
    \begin{column}{0.35\textwidth}
      \begin{equation*}
        \xymatrix{
          F(A) & F(B) \ar[l]_{F(f)} \\
          & F(C) \ar[u]_{F(g)} \ar[ul]^{F(h)}
        }    
      \end{equation*}
    \end{column}
  \end{columns}

  \begin{block}{Équivalence, dualité}
    \begin{itemize}
    \item Équivalence si $F:\C\ra\D$ et $G:\D\ra\C$ covariants 
    \item Dualité si $F:\C\ra\D$ et $G:\D\ra\C$ contravariants 
    \end{itemize}
    et $\quad F\comp G \simeq \Id_\D\quad$ et $\quad G\comp F \simeq \Id_\C$.
  \end{block}
  
\end{frame}


\begin{frame}
  \frametitle{Le principe de Tellegen}

  \Large
  \begin{center}
    ``De tout \alert{\emph{algorithme linéaire}} qui calcule une
    application linéaire on peut déduire an autre
    \alert{\emph{algorithme linéaire}} qui calcule la transposée de
    l'application ayant \emph{à peu près} les mêmes complexités en
    espace et temps.''
  \end{center}
\end{frame}


\begin{frame}[fragile]
  \frametitle{Un exemple}

  \begin{columns}

    \begin{column}{0.5\textwidth}
      \begin{center}
        \begin{minipage}{0.7\textwidth}
\begin{semiverbatim}
  for i = 1 to n-2 do
    a[i+1] = a[i] + a[i+1]
    a[i] = 0
  end for
\end{semiverbatim}
        \end{minipage}
      \end{center}
    \end{column}

    \begin{column}{0.5\textwidth}

      \begin{equation*}
        \begin{pmatrix}
          0 & \hdotsfor{3} & 0\\
          \vdots  &  &\vdots&& \vdots \\
          0 & \hdotsfor{3} & 0\\
          1 & \hdotsfor{3} & 1
        \end{pmatrix}
      \end{equation*}

    \end{column}
  \end{columns}
\end{frame}


\begin{frame}
  \frametitle{Calculs}

  \begin{block}{Langage, taille}
    \begin{itemize}
    \item Ensemble d'\emph{instructions} \hfill $\L \subset \hom(\C)$,
    \item Fonction de taille \hfill $\size{.} : \ob(\C)\ra\N$.
    \end{itemize}
  \end{block}

  \begin{block}{Calcul}
    \begin{center}
      Suite $\quad C_1:b\ra c\quad$ d'instructions.
    \end{center}
  \end{block}

  \[\xymatrix{
    & \ar[r]\ar@{}[dr]|{C_2} & \ar[dr] &
    & \ar[r]\ar@{}[dr]|{C_1} & \ar[dr] \\
    a \ar[ur]\ar@{-->}[rrr]_{f_2} &&& b
    \ar[ur]\ar@{-->}[rrr]_{f_1} &&& c
  }\]

  \begin{block}{Coûts en temps et espace}
    \begin{itemize}
    \item $\Time(C) = $ longueur du calcul,
    \item $\Space(C) = \max_{o\in C}\size{o}$.
    \end{itemize}
  \end{block}

\end{frame}


\begin{frame}
  \frametitle{Le cas de $\lcat{FMod}_R$}
  A.~Bostan, G.~Lecerf, E.~Schost :

  \begin{align*}
    +_1 : R^2 &\ra R^2         &   +_2 : R^2&\ra R^2       &  *_a : R&\ra R\\
    (p,q)&\mapsto(p+q,q)  &      (p,q)&\mapsto(p,p+q) &       p&\mapsto ap
  \end{align*}
  \begin{align*}
    \pi : R&\ra 0     &  \iota : 0&\ra R   \\
          p&\mapsto0  &          0&\mapsto0
  \end{align*}
  
  \begin{block}{}
    \begin{equation*}
      \label{eq:linlang}
      \L = \Bigl\{\Id_n\times\mathrm{op}\times\Id_m \Bigl\lvert
      n,m\in\N, \mathrm{op}\in\{+_1,+_2,*_a,\pi,\iota\;|\;a\in R\} \Bigr\}
      \text{.}
    \end{equation*}
  \end{block}

  \begin{block}{}
    \begin{center}
      $\size{R^n} = n$
    \end{center}
  \end{block}

\end{frame}


\begin{frame}[fragile]
  \frametitle{Notre exemple}

  \begin{columns}

    \begin{column}{0.5\textwidth}
      \begin{center}
        \begin{minipage}{0.7\textwidth}
\begin{semiverbatim}
  \alert<2>{for i = 1 to n-2 do}
    a[i+1] = a[i] + a[i+1]
    a[i] = 0
  end for
\end{semiverbatim}
        \end{minipage}
      \end{center}
    \end{column}

    \begin{column}{0.5\textwidth}
      \begin{minipage}{0.7\textwidth}
        \begin{center}
          $\phantom{ }$\\
          $\Id_{i}\times +_2 \times\Id_{n-2-i}$\\
          $\Id_i\times *_0 \times\Id_{n-1-i}$\\
        \end{center}
      \end{minipage}
    \end{column}
  \end{columns}

  \vfill

  \begin{center}
    \[\xymatrix{
      R^n\ar[r]^{+_2\times{\Id_{n-2}}} & R^n\ar[r]^{\*_0\times{\Id_{n-1}}} &
      R^{n} \ar@{.}[rr] && R^n
    }\]
  \end{center}
\end{frame}


\begin{frame}[fragile]
  \frametitle{Branchements}

  \begin{center}
    \[\xymatrix{
             &        &                 & \ar@{.}[r] & \\
      \ar[r] & \ar[r] & \ar[ru] \ar[rd] \\
             &        &                 & \ar@{.}[r] & 
    }\]
  \end{center}

  \begin{center}
    \begin{minipage}{0.7\textwidth}
\begin{semiverbatim}
  \alert<2>{if \alt<1>{a = (0,...,0)}{n = 0} then}
    ...
  else
    ...
  endif
\end{semiverbatim}
    \end{minipage}
  \end{center}
  
\end{frame}


\begin{frame}
  \frametitle{Algorithme}

  \begin{block}{Espace des paramètres}
    \begin{center}
      $\Par$ un ensemble récursivement énumérable
    \end{center}
    \begin{center}
      Par exemple, $\quad \Par=\N$
    \end{center}
  \end{block}

  \begin{block}{Algorithme}
    \begin{center}
      Une fonction $\qquad A:\Par \ra \C_\ra \qquad$ ($\C_\ra =$ les calculs)
    \end{center}
  \end{block}

  \begin{columns}

    \begin{column}{0.1\textwidth}
      \begin{align*}
        \uncover<2->{1}\\
        \uncover<3->{2}\\
        \uncover<4->{3}\\
        \uncover<4->{\vdots}
      \end{align*}
    \end{column}
    \begin{column}{0.9\textwidth}
      \begin{equation*}
        \xymatrix{
          \only<2->{\ar[r]} & \only<2->{\ar[rd]} & & \only<3->{\ar[r]}&\\
                            &                    & \only<2->{\ar[r]} \only<3->{\ar[ru]}&\\
          \only<3>{\ar[r]} \only<4->{\ar@{=>}[r]} & \only<3->{\ar[ur]}\only<4->{\ar[r]} & \only<4->{\ar[r]} & \only<4->{\ar[r]} &
        }
      \end{equation*}
    \end{column}
  \end{columns}
  
\end{frame}


\begin{frame}
  \frametitle{Complexité}

  \begin{block}{Complexité en temps}
    \begin{center}
      $A:\Par\ra\C_\ra\qquad$ induit une fonction
      $\qquad\Time_A:\Par\N\qquad$ donnée par
      \[\Time_A(x) = \Time(A(x))\]
    \end{center}
  \end{block}

  \begin{block}{Complexité en espace}
    \begin{center}
      $A:\Par\ra\C_\ra\qquad$ induit une fonction
      $\qquad\Space_A:\Par\N\qquad$ donnée par
      \[\Space_A(x) = \Space(A(x))\]
    \end{center}
  \end{block}
\end{frame}



\begin{frame}[fragile]
  \frametitle{Notre exemple}

  \begin{center}
    \begin{minipage}{0.7\textwidth}
\begin{semiverbatim}
\only<1>{  for i = 1 to n-2 do
    a[i+1] = a[i] + a[i+1]
    a[i] = 0
  end for}
\only<2>{  a[1] = a[0] + a[1]
  a[0] = 0
  a[2] = a[1] + a[2]
  a[1] = 0
  ...
  a[n-1] = a[n-2] + a[n-1]
  a[n-2] = 0}
\end{semiverbatim}
    \end{minipage}
  \end{center}

  \vfill

  \begin{center}
    \[\uncover<2>{n \quad\mapsto\quad}\xymatrix{
      R^n\ar[r]^{+_2\times{\Id_{n-2}}} & R^n\ar[r]^{\*_0\times{\Id_{n-1}}} &
      R^{n} \ar@{.}[rr] && R^n
    }\]
  \end{center}
\end{frame}


\begin{frame}
  \frametitle{Le théorème de Tellegen}

  \begin{block}{Foncteur de Tellegen}
    Un foncteur $\qquad F:\C\ra\D \qquad$ est dit de Tellegen si
    $F(\L_\C) \subset \L_\D$.
  \end{block}

  \begin{block}{Théorème de Tellegen}
    \begin{itemize}
    \item $F:\C\ra\D$ un foncteur de Tellegen
    \item $\Par$ un espace de paramètre
    \item $A:\Par\ra\C_\ra$ un algorithme
    \end{itemize}

    $F\comp A$, noté $F(A)$ est un algorithme de $\Par\ra\D_\ra$ avec
    \begin{itemize}
    \item $\Time_{F(A)} = \Time_a$,
    \item $\Space_{F(A)} = B(\Space_A)$ si $B:\N\ra\N$ est une borne sur $F$.
    \end{itemize}
  \end{block}
\end{frame}


\begin{frame}
  \frametitle{Le cas $\lcat{FMod}_R$}
  
  \begin{center}
    On connaît un foncteur (contravariant) $\quad
    T:\lcat{FMod}_R\ra\lcat{FMod}_R\quad$ donné par la transposition
    des matrices.
  \end{center}

  \begin{block}{Théorème de Tellegen pour l'algèbre linéaire}
    $T$ est un foncteur de Tellegen pour le langage $\L$ donné auparavant.
  \end{block}

  \begin{align*}
    T(+_1) = +_2 \qquad T(*_a) = *_a \qquad T(\pi) = \iota
  \end{align*}
\end{frame}



\begin{frame}[fragile]
  \frametitle{Notre exemple}

  \begin{columns}
    \begin{column}{0.5\textwidth}
      \begin{center}
        \begin{minipage}{0.7\textwidth}
\begin{semiverbatim}
  a[1] = a[0] + a[1]
  a[0] = 0
  a[2] = a[1] + a[2]
  a[1] = 0
  ...
  a[n-1] = a[n-2] + a[n-1]
  a[n-2] = 0
\end{semiverbatim}
        \end{minipage}
      \end{center}
    \end{column}

    \begin{column}{0.5\textwidth}
      \begin{center}
        \begin{minipage}{0.7\textwidth}
\begin{semiverbatim}
  a[n-2] = 0
  a[n-2] = a[n-2] + a[n-1]
  ...
  a[1] = 0
  a[1] = a[1] + a[2]
  a[0] = 0
  a[0] = a[0] + a[1]
\end{semiverbatim}
        \end{minipage}
      \end{center}
    \end{column}
    \end{columns}
  
  \vfill
  \pause

  \begin{columns}

    \begin{column}{0.5\textwidth}
      \begin{center}
        \begin{minipage}{0.7\textwidth}
\begin{semiverbatim}
  for i = n-2 to 0 do
    a[i] = 0
    a[i] = a[i] + a[i+1]
  end for
\end{semiverbatim}
        \end{minipage}
      \end{center}
    \end{column}

    \begin{column}{0.5\textwidth}

      \begin{equation*}
        \begin{pmatrix}
          0 & \hdotsfor{3} & 0\\
          \vdots  &  &\vdots&& \vdots \\
          0 & \hdotsfor{3} & 0\\
          1 & \hdotsfor{3} & 1
        \end{pmatrix}
      \end{equation*}

    \end{column}
  \end{columns}
\end{frame}

\end{document}


% Local Variables:
% mode:flyspell
% ispell-local-dictionary:"francais"
% End:

% LocalWords:  Tellegen
